\documentclass{article}
% \documentclass[twocolumn]{article}
\usepackage[a4paper,margin=2cm]{geometry}
\usepackage{lipsum,amsmath}

\def\QC{\textbf{QC} }

\title{A (hybrid) quantum SAT solver}
\author{Henri de Boutray}
\date{\today}

\begin{document}

\maketitle

\begin{abstract}
Quantum computing (\QC) experts has proven may times that \QC may bring great
improvement in respect of computational speed. But most potential clients, as of
today, only heard the buzz words, without seeing what \QC was capable of. This
is for a good reason: today, the actual quantum computers are barely reaching a
level where they are relevant. This is not to say that the promises are in vain
though, simply that Rome wasn't built in one day: neither is the quantum
computer. This is why the business plan of ColibrITD cannot be to ``simply''
build a revolutionary platform: we need to have convincing arguments for clients
to use quantum. For this reason, we are looking for good use-cases. In addition
to being good demos, they can be part of our grand plan: building a platform for
companies to use \QC without needing the deep technical knowledge usually
required. Such use-cases were previously established: a Navier-Stokes equation
solver, an optimizer, a compiler, \emph{etc...} I will here present an idea of
quantum SAT solver, using the power of superposition in order to find solution
to systems of boolean equations.
\end{abstract}

\begin{center}
\emph{
\textbf{Keywords:} SAT problem, combinatorial exploration, quantum computing,
quantum speedup
}
\end{center}

\section{Why: Solving problems with SAT}
\label{sec:solving_problems_with_sat}

Problem solving is a cornerstone of our society. Improving processes should not
be a end in itself, but it is definitely a mean to improve the lives of each of
us. Amongst the problems we are confronted to everyday (conscientiously or not),
discrete problems are omnipresent. This is why I chose to focus on those in this
article. In particular, on a specific class of those problems called SAT
problems.

SAT or boolean SATisfiability problems are problems encoded as a logic formula,
and they enable the solving of many other discrete problems. First of all, SAT is
a very complex problem to solve (called a NP-complete problem) \cite{Coo71}, 
which makes it a great theoretical study subject to analyze fundamental notions
concerning quantum advantage (in particular the likely differences between the P,
NP and quantum equivalent classes of complexity). But they are much more than
that.

Most discrete optimization problems can be encoded as SAT problems, which make
the SAT solvers extremely useful tools. Here are some example of areas where SAT
solvers are used:
\begin{itemize}
  \item automated testing (model checking), for software but not only 
    \cite{SBS96,BCCZ99};
  \item automated/semi-automated theorem proving and formal software verification
    \cite{DLL62,Sta94,BFMP11};
  \item Electrical Design Automation (EDA):
  \begin{itemize}
    \item formal equivalence checking \cite{PK00};
    \item FPGA routing \cite{NSR02};
  \end{itemize}
  \item cryptoanalysis \cite{TID20};
  \item various discrete optimization problems \cite{LAK+14}.
\end{itemize}
These problems are usually huge in term of number of variables and formula size,
which implies an important need of efficient solvers.

Solving more efficiently those problems using quantum computing would then enable
us to use the quantum advantage for many discrete problems. We could then sell
our quantum SAT solver or use the SaaS approach and sell our expertise by helping
companies to efficiently solve their discrete problems using quantum computers.

\section{How: Good candidates for quantum computing}
\label{sec:good_candidates_for_quantum_computing}

SAT problems are especially interesting because of the huge size of the problems.
Indeed, \QC is all about gaining a complexity advantage over classical computing.
Many theoretical example examples have already been exhibited (more than 400
references split into more that 60 categories of algorithms centralized in the
quantum algorithm zoo \cite{Jor21}), some of them have been run on quantum
processors, but the real promise is still just that: a promise. Indeed, the the
nature of complexity comparison, the quantum advantage only gets bigger as the
size of the systems grow. This is the reason why SAT problems are such good
candidate: the real life example are typically using several thousand of
variables and exponentially more clauses. They are easy to check but because of
the number of variables, it is extremely long to brute-force the solution.

A SAT problem is most often given as a logic formula in the Conjunctive Normal
Form (CNF). A CNF is conjunction of clauses, \emph{i.e.} a set of clauses that
must all be true conjointly (at the same time) for the. These clauses are
themselves disjunction of literals, sets of literals that can be disjointly true
for the whole formula to be true. Finally, each literal is either a variable or
the negation of a variable. Concretely, a CNF would look like the formula shown
in Eq.\ref{eq:cnf}.
\begin{equation}
\label{eq:cnf}
\bigwedge_{i=1}^n \left(\bigvee_{j=1}^m l_{i,j}\right) \text{ where } l_{i,j} \in
\{v_{i,j},\neg v_{i,j}\}
\end{equation}

The SAT problems being in the NP-complete complexity class, they are classically
hard to solve. A brute-force search requires going through $2^n$ elements, if
there are $n$ variables. Since there are often around $10^3$ variables for this
king of problems, $2^{1000}$ solutions need to be examined. With current rate of
computing (exascale, $10^{18}$ operations per second \cite{Hin18}), this would
take around $10^{275}$ years. Using a quantum search algorithm may transform
these questions from completely impossible to possible.

\section{What: The known the plan and the promising}
\label{sec:the_known_the_plan_and_the_promising}

As they often are, this previous short presentation of a complex subject was a
bit reductive. Indeed, as said previously, SAT solvers are used in day to day
life, so SAT problems are solvable in less that $10^{275}$ years. Smarter than
brute-force solutions exist, they are not guarantied to finish in a reasonable
time so they are most often equipped with a timeout kill switch. These solvers
are estimated to solve a problem in a time in $O(1.329\ldots^n)$. 

In comparison, if we manage to create an oracle for this problem, we could use
Grover's algorithm \cite{Gro96} to solve it with a complexity in $O(\sqrt{N})=O
(1.41\ldots^n)$ (the system has $n$ qubits and $N=2^n$ base states). And here is
the first speed bump, as discussed in \cite{Amb05}, Grover's algorithm alone in
not sufficient to leverage a quantum advantage over the already existing
algorithms that are cores to SAT solvers. This is not the end of the line
though, by combining the current classical algorithms with Grover's, we could
obtain a quadratic acceleration, having a complexity in $O(1.153\ldots^n)$.

Other leads have also been explored: by nesting quantum search, Cerf \emph{et
al.} managed to gain an exponential speedup \cite{CGW00}, and by using a very
different method all together, Bian \emph{et al.} manages to really recently use
a Quantum Annealer to explore the space of possibilities by encoding the problem
as an \emph{Izing model} characterizing the quantum annealer \cite{BCM+20}.

All these options are promising. They may have an important time cost for us, as
SAT solvers are very specialized pieces of softwares, and their intersection
with other problems may be small. But because of the remarkable possibilities
they would open, being the first to have a working quantum SAT solver may put us
in a very good position.

We should also keep in mind that de Beaudrap \emph{et al.} managed to encode SAT
problems into ZH-calculus --a variant of the diagrammatic quantum language
ZX-calculus-- and use its rewrite rules to give tools to simplify SAT formulas
\cite{dKM21}. This may be an interesting lead as it means that SAT formulas may
be encoded as quantum circuits, but how to use those circuits to solve the SAT
problems is left to be determined.

\bibliographystyle{alpha}
\bibliography{SAT.bib}

\end{document}